% This is how we did it

To enable production of useful data, our case studies required a model wind farm with characteristics that were simultaneously restrictive enough to maintain simplicity, yet general enough to maintain relevance to more complex and realistic problems.
The wind farm scenarios we selected to meet this criteria, and other details relevant to this project as a whole, are described below in \cref{sec:windfarm}.

Many factors affect recommendations for superior turbine placement of a proposed wind farm.
The two major factors we chose to study are
1) EWM characteristics and 2) optimization algorithm \cite{HerbertAcero2014}.
We designed two case studies in an attempt to quantify the effects of each of these choices.

For the first case study, in which the goal was to isolate variability in the optimization method, we pre-coded a representative wake model as a control variable
and permitted participants to use any optimization strategy to alter turbine locations that would deliver the best annual energy production (AEP) for the farm.
This is called case study 1 and is described below in \cref{sec:optonly}.

Isolating EWM variability proved more complicated.
An EWM's compatibility with gradient-based or gradient-free optimization methods dictates which algorithms can be applied.
As such, designing a case study that restricted participants to a single optimization algorithm would unnecessarily limit the scope of EWMs studied.
For this reason, our second case study permitted not only participant selection of EWM but also the optimization algorithm.
It is called case study 2 and is described below in \cref{sec:cmbnd}.

\bigskip
\subsection{Common to Both Case Studies} \label{sec:windfarm}

	\import{./sections/}{iea37-wflocs-mthd-bothcs.tex}
	
\subsection{Case Study 1: Optimization Only} \label{sec:optonly}

	\import{./sections/}{iea37-wflocs-mthd-optonly.tex}

\subsection{Case Study 2: Combined Physics Model/Optimization Algorithm} \label{sec:cmbnd}

	\import{./sections/}{iea37-wflocs-mthd-cmbnd.tex}