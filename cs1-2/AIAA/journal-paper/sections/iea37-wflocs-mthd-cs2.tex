The intent of this case study was to assess both the effects that different optimization methods and physics model approximations have on turbine location recommendations.
Case study 2 differs from case study 1 in that 1) no wake model was provided and 2) only a single wind farm size was to be optimized.
Participants were free to choose their preferred EWM and optimization method combination.

\subsubsection{Wake Model}
Unlike case study 1, participant-reported AEP values were not comparable, since each participant used a different EWMs to calculate AEP. %(which account for different physics phenomena)
To help us make fair comparisons and conclusions, we conducted a cross-comparison of results between participants.
For the cross-comparison, each participant provided their optimal turbine layout in the standardized \texttt{.yaml} format. Each participant was then provided with every other participants' optimized layout file.
Participants then used their own wake model to calculate the AEP of the other participant's proposed farm layouts with their EWMs.
From this portion of the case study, we hoped to learn if any participants' results were consistently seen as superior by other EWMs.

\subsubsection{Farm Attributes}
The wind farm size for the combined case study was limited to nine turbines.
We did this to limit the computation time requirements when assessing results in a standardized LES, discussed later in \cref{sec:conc}.
We used the previously described method under \cref{sec:farmgeog} to determine the boundary radius, which for the 9 turbine case is 900 m.
The wind rose and wind speed were the same as in case study 1.