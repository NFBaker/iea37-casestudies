% This is what we did
% Overview
%	numbers
%	gradient/non gradient participants
% Winners
%	Best Gradients
%	Best non-Gradients
% Losers
%	Why it didn't work

\subsection{Case Study 1: Optimization Only}\label{sec:res-optonly}

	Each participant ran the optimization algorithm of their choosing using our suppied AEP target function, or a functional equivalent in another language.
	Since there exists a great deal of variability in hardware, participants also reported processor speed, function calls, number of cores utilized, and amount of RAM installed in their system when finding their optimized results.
	The AEP results and rankings are given below in \cref{tab:results1,tab:results2,tab:results3}.

	There were 10 submissions for the Optimization Only Case Study.
	One participant submitted twice, using a different optimization method for each submission.
	These two submissions are treated as if from different participants.
	For anonimity, each submission is assigned a number.
	We will refer to each submission below by this participant number (i.e. par1, ..., par10, etc.).

	\subsubsection{Data}

	\cref{tab:results1,tab:results2,tab:results3} display the final AEP data of all participant proposed optimal turbine layouts.
	The Python module we supplied which uses the simplified Gaussian wake model was used for all AEP calculations.
	Submissions are ranked from highest to lowest resultant AEP values, with participant number (par\#), whether using a gradient-based (Yes) or gradient-free (No) optimization method, and the percentage increase (Inc.) from the provided example layout's AEP.

		\begin{table}[H]
			\begin{center}
				\caption{16 turbine scenario participant results}
				\label{tab:results1}
				\begin{tabular}{r l l c r}
					\hline
					Rank           & AEP         & par\# & Grad. 	& Inc.\\
					\hline
					1              & 418924.4064 & 4     & Yes      & 14.17 \% \\
					2              & 414141.2938 & 5     & Yes      & 12.86 \% \\
					3              & 412251.1945 & 8     & Yes      & 12.35 \% \\
					4              & 411182.2200 & 1     & Yes      & 12.06 \% \\
					5              & 409689.4417 & 2     & No       & 11.65 \% \\
					6              & 408360.7813 & 10    & Yes      & 11.29 \% \\
					7              & 402318.7567 & 3     & No       &  9.64 \%\\
					8              & 392587.8580 & 7     & No       &  6.99 \%\\
					9              & 388758.3573 & 6     & No       &  5.95 \%\\
					10             & 388342.7004 & 9     & No       &  5.83 \%\\
					ex.			   & 366941.5712 & 11    & N/A      &  0.00 \%\\
					\hline
				\end{tabular}
			\end{center}
		\end{table}

	%\subsubsection{36 Turbine Case}

		\begin{table}[H]
			\begin{center}
				\caption{36 turbine scenario participant results}
				\label{tab:results2}
				\begin{tabular}{r l l c r}
					\hline
					Rank           	& AEP         	& par\# & Grad.	& Inc.\\
					\hline
					1				& 863676.2993	& 4		& Yes   & 17.95 \%\\
					2				& 851631.9310	& 10	& Yes   & 15.42 \%\\
					3				& 849369.7863	& 2     & No    & 15.11 \%\\
					4				& 846357.8142	& 8     & Yes   & 14.70 \%\\
					5				& 844281.1609	& 1     & Yes   & 14.42 \%\\
					6				& 828745.5992	& 3     & No    & 12.31 \%\\
					7				& 820394.2402	& 5     & Yes	& 11.18 \%\\
					8				& 813544.2105	& 9     & No    & 10.25 \%\\
					9				& 777475.7827	& 7     & No    &  5.37 \%\\
					10				& 776000.1425	& 6     & No    &  5.17 \%\\
					ex.				& 737883.0985	& 11    & N/A	&  0.00 \%\\
					\hline
				\end{tabular}
			\end{center}
		\end{table}

	%\subsubsection{64 Turbine Case}

		\begin{table}[H]
			\begin{center}
				\caption{64 turbine scenario participant results}
				\label{tab:results3}
				\begin{tabular}{r l l c r}
					\hline
					Rank        & AEP         	& par\# & Grad.	& Inc.\\
					\hline
					1			& 1513311.1936	& 4 	& Yes   & 16.86 \%\\
					2			& 1506388.4151	& 2		& No    & 16.33 \%\\
					3			& 1480850.9759	& 10	& Yes   & 14.35 \%\\
					4			& 1476689.6627 	& 1		& Yes   & 14.03 \%\\
					5			& 1455075.6084	& 3		& No    & 12.36 \%\\
					6			& 1445967.3772	& 8		& Yes   & 11.66 \%\\
					7			& 1422268.7144	& 9		& No    &  9.83 \%\\
					8			& 1364943.0077	& 6		& No    &  5.40 \%\\
					9			& 1336164.5498 	& 5		& Yes	&  3.18 \%\\
					10			& 1332883.4328	& 7		& No    &  2.93 \%\\
					ex.			& 1294974.2977	& 11	& N/A	&  0.00 \%\\
					\hline
				\end{tabular}
			\end{center}
		\end{table}

	\import{./sections/}{iea37-wflocs-rslts-optonly.tex}

\subsection{Case Study 2: Combined}\label{sec:res-cmbnd}

	For the Combined Wake Model and Optimization Method Seletion Case Study, each participant ran both the optimization algorithm of their choosing as well as their choce of wake model target function.
	There were no restrictions on programming language for either the wake model or optimization algorithm, but results of optimal turbine layouts were to be submitted in the \texttt{.yaml} format supplied in the Case Study 1 examples.

	There were 5 participant submissions for the Combined Case Study.
	All 5 also participated in Case Study 1 (though were not required to do so) so we assigned them the same participant numbers from that Case Study.
	i.e., \textit{par1} - \textit{par5} are the same for both Case Study 1 and Case Study 2.

	With each participant using a different wake model, AEP values reported by participants cannot be fairly compared.
	Results were judged on a cross-comparison of layouts between participants.

	\subsubsection{Data}

	The cross-comparison does displays some interesting trends.
	The following tables show how each participant's wake models ranked the proposed optimal turbine layouts for the other 4 participants.
	Each participant's ranking of their own layout is highlighted in \textbf{bold}.
	The last column in the table is the participant number of the layout being cross-compared (cc-par\#).
	So participant 4's analysis of participant 2's layout would be found in the par4 table, with 2 in the cc-par\# column.
	The last column is the percentage increase (Inc.) from the reporting participant's submitted layout.
	A negative value here indicates a worse AEP.

		\newpage
		\begin{table}[H]
			\begin{center}
				\begin{tabular}{r l r c r}
					par1 	& Rank	& AEP					& cc-par\#	& Inc.\\
					\hline
							& 1		& 262350.319			& 4 		& 0.624 \%\\	
							& 2		& 262282.416			& 5 		& 0.598 \%\\
							& 3		& \textbf{260722.295}	& \textbf{1}& 0.000 \%\\
							& 4		& 260640.906			& 3 		&-0.031 \%\\
							& 5		& 248215.024			& 2 		&-4.797 \%\\
					\hline
				\end{tabular}
			\end{center}
		\end{table}

		\begin{table}[H]
			\begin{center}
				\begin{tabular}{r l r c r}
					par2 	& Rank	& AEP					& cc-par\#	& Inc. \\
					\hline
							& 1		& 250464.9732			& 4 		& 5.975 \%\\	
							& 2		& 250249.0259			& 5 		& 5.884 \%\\
							& 3		& 247812.0522			& 3 		& 4.853 \%\\
							& 4		& 240309.5850			& 1 		& 1.678 \%\\
							& 5		& \textbf{236342.799}	& \textbf{2}& 0.000 \%\\
					\hline
				\end{tabular}
			\end{center}
		\end{table}

		\begin{table}[H]
			\begin{center}
				\begin{tabular}{r l r c r}
					par3 	& Rank	& AEP					& cc-par\#	& Inc.\\
					\hline
							& 1		& 247109.5234			& 5 		& 0.590 \%\\	
							& 2		& 246942.3767			& 4 		& 0.522 \%\\
							& 3		& \textbf{245659.4124}	& \textbf{3}& 0.000 \%\\
							& 4		& 242431.5431			& 2 		&-1.314 \%\\
							& 5		& 237548.6622			& 1 		&-3.302 \%\\
					\hline
				\end{tabular}
			\end{center}
		\end{table}

		\begin{table}[H]
			\begin{center}
				\begin{tabular}{r l r c r}
					par4 	& Rank	& AEP					& cc-par\#	& Inc.\\
					\hline
							& 1		& \textbf{257790.1924}	& \textbf{4}& 0.000 \%\\	
							& 2		& 257663.4068			& 5 		&-0.049 \%\\
							& 3		& 255063.8201			& 3 		&-1.058 \%\\
							& 4		& 251776.7157			& 1 		&-2.333 \%\\
							& 5		& 239612.8223			& 2 		&-7.051 \%\\
					\hline
				\end{tabular}
			\end{center}
		\end{table}

		\begin{table}[H]
			\begin{center}
				\begin{tabular}{r l r c r}
					par5 	& Rank	& AEP					& cc-par\#	& Inc.\\
					\hline
							& 1		& \textbf{251771.9067}	& \textbf{5}& 0.000 \%\\	
							& 2		& 251697.7126			& 4 		&-0.029 \%\\
							& 3		& 249829.2199			& 3 		&-0.772 \%\\
							& 4		& 246503.8323			& 1 		&-2.092 \%\\
							& 5		& 239482.6767			& 2 		&-4.881 \%\\
					\hline
				\end{tabular}
			\end{center}
		\end{table}

	\import{./sections/}{iea37-wflocs-rslts-cmbnd.tex}

%---- Eduardo's stuff --%
%As this research progresses, a validation of the propeller-on-propeller interactions predicted by VPM will be performed in three phases:

%\begin{enumerate}
%	\item Modeling of the exact experimental setup used in the PIV measurements reported by Zhou \textit{et al.}\cite{Zhou2017} and comparison of the predicted velocity field of this two co-rotating propellers.
%	\item Sweeping of separation distance between the two co-rotating propellers used by Zhou \textit{et al.} and comparison between measured and predicted aerodynamic performance (thrust and torque).
%	\item Once validity is established, we will perform a parametric study of performance on APC 10x7 propellers interacting at varying advance ratios, Reynolds numbers, and separation distance, in counter and co-rotation configurations.
%\end{enumerate}

%With the development and validation of the method presented in this study we aim to show the capabilities of the VPM to model propeller-on-propeller interactions in a first-principles-based approach, with an accuracy and speed well fit for the conceptual design of distributed-propulsion aircraft.