With the calculations for AEP standardized, each participant ran the optimization algorithm and implementation of their choosing.
Since there exists a great deal of variability in hardware, participants also reported processor speed, function calls, number of cores utilized, and amount of RAM installed the system utilized to find their optimized results.


\subsection{Case Study 1: Optimization Only}

\subsubsection{16 Turbine Case}

	\begin{center}
		\begin{tabular}{r l r}
			\hline
			Participant \# & AEP & Rank \\
			\hline
			example layout & 366941.5712 & 11 \\
			1 & 411182.2200 & 4 \\
			2 & 409689.4417 & 5 \\
			3 & 402318.7567 & 7 \\
			4 & 418924.4064 & 1 \\
			5 & 414141.2938 & 3 \\
			6 & 388758.3573 & 9 \\
			7 & 392587.8580 & 8 \\
			8 & 412251.1945 & 2 \\
			9 & 388342.7004 & 10 \\
			10 & 408360.7813 & 6 \\
			\hline
		\end{tabular}
	\end{center}



\subsubsection{36 Turbine Case}

\subsubsection{64 Turbine Case}


\subsection{Case Study 2: Combined}

Not enough time/resources for LES right now.

Cross-Comparison of results was conducted.

%---- Eduardo's stuff --%
%As this research progresses, a validation of the propeller-on-propeller interactions predicted by VPM will be performed in three phases:

%\begin{enumerate}
%	\item Modeling of the exact experimental setup used in the PIV measurements reported by Zhou \textit{et al.}\cite{Zhou2017} and comparison of the predicted velocity field of this two co-rotating propellers.
%	\item Sweeping of separation distance between the two co-rotating propellers used by Zhou \textit{et al.} and comparison between measured and predicted aerodynamic performance (thrust and torque).
%	\item Once validity is established, we will perform a parametric study of performance on APC 10x7 propellers interacting at varying advance ratios, Reynolds numbers, and separation distance, in counter and co-rotation configurations.
%\end{enumerate}

%With the development and validation of the method presented in this study we aim to show the capabilities of the VPM to model propeller-on-propeller interactions in a first-principles-based approach, with an accuracy and speed well fit for the conceptual design of distributed-propulsion aircraft.