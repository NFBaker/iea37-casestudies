\lettrine[nindent=0pt]{O}{ptimizing} turbine placement within a wind farm is a complex problem characterized by many local minima.
The large number of inter-dependent variables involved in Wind Farm Layout Optimization (WFLO) create a design space that can quickly become intractable.
%The two main methods to attack WFLO are (1) utilization of engineering wake model and (2) computational optimization to iteratively discover an optimal set of inputs.
In this study, we designed and conducted a set of case studies to discover the best practices in solving the WFLO problem.

Two approaches have been taken to simplify the WFLO problem, as described by Padron et.al.\cite{Padron2018}
The first approach aims at improving the quality of individual models for wind farm attributes, i.e., aerodynamics, atmospheric physics, turbine structures, etc.
The second approach is to improve formulating the optimization problem, as well as the algorithms used to solve the optimization. \cite{Padron2018}

To better model the aerodynamics of the waked airflow region of a wind turbine, complex computational methods such as such as Direct Numerical Simulations (DNS) or Large Eddy Simulations (LES) have been developed.
But the computational time these require for full simulation can be prohibitive in multi-iterative optimizations.
Simplified Engineering Wake Models (EWMs) make certain limiting physics assumptions, resulting in greatly reduced computational costs. \cite{HerbertAcero2014}.
Yet these simpler, less accurate approximations can sometimes lead to inefficient recommendations for turbine placement, due to what can be inaccurate assumptions in specific wind farm scenarios. %shorten to a single sentence

Given a single EWM, optimization methods to select ideal turbine locations are limited by characteristics of the functions governing the model.
For example, EWMs that define a discontinuous wind speed behind wind turbines cannot be effectively used with gradient-based optimization methods, and models for which gradients have not been calculated are limited to gradient-free algorithms, or gradient-based with finite difference derivatives.
Additionally, within these limitations different optimization strategies have varying capacity to escape local optima in the pursuit of a global optimum. %primarily by the ability to obtain gradients. I don't think this is necessarily true, I'll reword

To better understand the differences in EMW selection and optimization algorithm application, we have created two case studies.
These studies are designed to involve participants from many different research labs, working on both general optimization methods, as well as methods specific to solving the WFLO problem.
The first isolates optimization techniques for a single simplified EWM, the second observes the differences when combining variations in EWM selection and optimization method.

In our research, we have not found case studies of a similar degree previously conducted.
Though papers have been published which survey the state of the Wind Farm Optimization (perhaps most notibly a paper by Herbert and Acero \cite{HerbertAcero2014}),
our case studies are the first time international collaboration has been conducted to isolate optimization method and EWM selection to determine best practices when addressing the WFLO problem.

%This work is done in support of the International Energy Agency (IEA)'s Task 37.
%The IEA was created in 1974, and currently has 30 member countries.
%Its mission is ``to ensure reliable, affordable and clean energy''\cite{IEAwebsite} for those countries, and does so through four areas of focus:

%    \begin{itemize}
%        \item Energy security
%        \vspace{-3mm}
%        \item Economic development
%        \vspace{-3mm}
%        \item Environmental awareness
%        \vspace{-3mm}
%        \item Engagement worldwide.
%    \end{itemize}
%The IEA's Technology Collaboration Program (TCP) has a Working Party on Renewable Energy Technologies (REWP).
%REWP, itself, has a Wind Energy TCP, which is further subdivided into numbered tasks.
%These tasks cover individual concepts relative to wind energy\cite{ieawind}.
%For example Task 19 deals with Wind Energy in Cold Climates, Task 26 deals with the Cost of Wind Technology.

Our case studies are created in support of the International Energy Agency (IEA)'s Wind Task 37 (IEA37).
IEA37 coordinates international research activities centered around the analysis of wind power plants as holistic systems\cite{IEATask372017}.
Though our case studies concentrate mainly on wake modelling optimization at the farm-level scale, our results still contribute to what IEA37 terms a ``hollistic'' approach\cite{IEATask372017} to wind energy.