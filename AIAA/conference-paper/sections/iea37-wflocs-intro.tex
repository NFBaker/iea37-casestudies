\lettrine[nindent=0pt]{O}{ptimizing} turbine placement within a wind farm is a complex problem characterized by many local minima.
The large number of inter-dependent variables involved in Wind Farm Layout Optimization (WFLO) create a design space that can quickly become intractable.
%The two main methods to attack WFLO are (1) utilization of engineering wake model and (2) computational optimization to iteratively discover an optimal set of inputs.

Two approaches have been taken to simplify the WFLO problem, as described by Padron et.al.\cite{Padron2018} The first approach aims at improving the quality of individual models for wind farm attributes, i.e., aerodynamics, atmospheric physics, turbine structures, etc.
As they go on to state, "The second approach is to improve the optimization problem formulation, and the algorithms [used] to solve the optimization." \cite{Padron2018}

To better model the aerodynamics of the waked airflow region of a wind turbine, complex computational methods such as such as Direct Numerical Simulations (DNS) or Large Eddy Simulations (LES) have been developed.
But the computational time these require for full simulation can be prohibitive in multi-iterative optimizations.
Simplified Engineering Wake Models (EWMs) make certain limiting physics assumptions, resulting in greatly reduced computational costs. \cite{HerbertAcero2014}.
Yet these simpler, less accurate approximations can sometimes lead to inefficient recommendations for turbine placement, due to what can be inaccurate assumptions in specific wind farm scenarios. %shorten to a single sentence

Given a single EWM, optimization methods to select ideal turbine locations are limited by characteristics of the functions governing the model.
For example, EWMs that define a discontinuous wind speed behind wind turbines cannot be effectively used with gradient-based optimization methods, and models for which gradients have not been calculated are limited to gradient-free algorithms or gradient-based with finite difference derivatives.
Additionally, within these limitations different optimization strategies have varying capacity to escape local minima in the pursuit of a global optimum. %primarily by the ability to obtain gradients. I don't think this is necessarily true, I'll reword

To better understand the differences in EMW selection and optimization algorithm application, we have created two discrete case studies.
These studies are designed to involve participants from many different research labs working on the WFLO problem.
The first isolates optimization techniques for a single simplified EWM, the second observes the differences when combining variations in EWM selection and optimization method.

\todo[inline, color=red!40]{Explain how people haven't done this before}
\todo[inline, color=red!40]{History/bio of IEA}
\todo[inline, color=red!40]{history/bio of Task37}
\todo[inline, color=red!40]{History/bio of NREL}