The purpose of this Case Study is to determine the best optimization practices for WFLO, using a single representative EWM.
    
    \subsubsection{Wake Model} \label{sec:wakemodel}
        A simplified version of Bastankhah's Gaussian wake model \cite{Bastankhah2016, Thomas2018} is used, since it is compatible with both gradient-based and gradient-free methods, and is computationally inexpensive in comparison to LES and DNS methods.
        This wake model is described by the following equations:

        \begin{equation}
            \frac{\Delta U}{U_{\infty}}
            =
            \Bigg(
            1 - \sqrt{
                1 - \frac{C_T}
                {8\sigma_{y}^{2}/D^2}
            }
            \Bigg)
            \text{exp}\bigg(
            -0.5\Big(
                \frac{y-\delta}{\sigma_{y}}
                \Big)^2
            \bigg)
            \label{eq:wakemodel}
        \end{equation}

        \noindent Where $\frac{\Delta U}{U_{\infty}}$ is the wake velocity deficit, $C_T = \frac{8}{9}$ and is the thrust coefficient, $y-\delta$ is the distance of the point of interest from the wake center in the cross-stream horizontal direction, $D$ is the turbine diameter, and $\sigma_y$ is the standard deviation of the wake deficit in the cross-stream horizontal direction as defined in \cref{eq:sigy}:

        \begin{equation}
            \sigma_y = (k_y \cdot x) + \frac{D}{\sqrt{8}} \\
            \label{eq:sigy}
        \end{equation}

        In \cref{eq:sigy}, $x$ is the downstream distance from the turbine generating the wake to the turbine of interest, and $D$ is the turbine diameter. $k_y$ is determined as a function of turbulence intensity ($I$).
        In this case study turbulence intensity is treated as a very small constant of $0.075$, and we therefore used a coresponding $k_{y}$ of $0.0324555$ \cite{Niayifar2016, Thomas2018}.

        Increasing turbulence intensity has numerous effects and draws attention away from the main purpose of this Case Study, which is to observe the differences of optimization strategies.
        For the wake model we use (given in \cref{eq:wakemodel}), increasing the turbulence intensity widened the wake cone, but second and third order effects are unknown.
        As such, this first IEA Task 37 set of Case Studies uses a very low intensity in an attempt to minimize the considered variables.

	\subsubsection{Farm Sizes}
        Variability in wind farm size (and thus number of design variables) affect optimization algorithm performance.
        To account for this, 3 wind farm sizes are specified in Case Study 1: 16, 36, and 64 turbines.
        These had farm boundary radii of 1300 m, 2000 m, and 3000 m respectively, determined in the manner described previously in \cref{sec:farmgeog}.
        Inclusion of 3 farm sizes is to observe how increased complexity correlates to convergence time and algorithm performance, to demonstrate trends of scalability in optimization algorithms.
        
        The turbine numbers are selected as perfect squares which roughly double in size.
        Perfect squares are used to permit even grid turbine arrangements, if desired.
        
    \subsubsection{Supplied Code} \label{sec:code}

        To enable participation in this Case Study, we created and supplied a pre-coded Python module.
        This module includes:
            \begin{itemize}
                \item Turbine charactersitics, wind frequency, and wind speed in IEA 37 WP X's \texttt{.yaml} schema
                \item Example turbine layouts for each farm size (in \texttt{.yaml} format)
                \item Python parsers of the \texttt{.yaml} schema
                \item Python target function to calculate AEP (given \texttt{.yaml} turbine locations and farm attributes)
            \end{itemize}

        We selected the programming language Python since it is open source and widely used by researchers in the industry.

        Participant alteration to our specific code implementation, or replication of our model in another language, was permitted if needed for compatibility with participant optimization methods.
        This is with the understanding, however, that final wind farm layouts would be evaluated with the original Python code that we provided.