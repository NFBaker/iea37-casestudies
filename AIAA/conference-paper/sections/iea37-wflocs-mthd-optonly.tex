The purpose of this Case Study is to determine the best optimization practices for WFLO, using a single representative EWM.
    
    \subsubsection{Wake Model} \label{sec:wakemodel}
        A simplified version of Bastankhah's Gaussian wake model \cite{Bastankhah2016, Thomas2018} is used, since it is compatible with both gradient-based and gradient-free methods, and is computationally inexpensive in comparison to LES and DNS methods.
        This wake model is described by the following equations:

        \begin{equation}
            \frac{\Delta U}{U_{\infty}}
            =
            \Bigg(
            1 - \sqrt{
                1 - \frac{C_T}
                {8\sigma_{y}^{2}/D^2}
            }
            \Bigg)
            \text{exp}\bigg(
            -0.5\Big(
                \frac{y-\delta}{\sigma_{y}}
                \Big)^2
            \bigg)
        \end{equation}

        \noindent Where $\frac{\Delta U}{U_{\infty}}$ is the wake velocity deficit, $C_T = \frac{8}{9}$ and is the thrust coefficient, $y-\delta$ is the distance of the point of interest from the wake center in the cross-stream horizontal direction, $D$ is the turbine diameter, and $\sigma_y$ is the standard deviation of the wake deficit in the cross-stream horizontal direction as defined in \cref{eq:sigy}:

        \begin{equation}
            \sigma_y = k_y(x) + \frac{D}{\sqrt{8}} \\
            \label{eq:sigy}
        \end{equation}

        In \cref{eq:sigy}, $x$ is the downstream distance from the turbine generating the wake to the turbine of interest, and $D$ is the turbine diameter. $k_y$ is determined as a function of turbulence intensity ($I$).
        In this case study turbulence intensity is treated as a very small constant of $0.075$, and we therefore use a coresponding $k_{y}$ of $0.0324555$ \cite{Niayifar2016, ThomasBast2018}.

        Increasing turbulence intensity has numerous effects and draws attention away from the main purpose of this Case Study, which is to observe the differences of optimization strategies.
        Though these strategies may differ for a wake model with a higher turbulence intensity, this first IEA Task 37 set of Case Studies uses a very low intensity in an attempt to minimize the considered variables.

	\subsubsection{Farm Sizes}
        Variability in wind farm size effect optimization algorithm performance.
        To account for this, 3 wind farm sizes are specified in Case Study 1: 16, 36, and 64 turbines.
        Inclusion of 3 farm sizes is to avoid a bias towards algorithms optimized for wind farms of a specific size, and in order to observe how increased complexity correlates to convergence time and algorithm performance.
        
        The turbine numbers were selected as perfect squares which roughly double in size.
        Perfect squares are selected to permit even grid turbine arrangements, if desired.

        Example \texttt{.yaml} schema is provided to all participants, to help understand the format with which they will need to report their optimal turbine locations.
        These files are:

        \begin{itemize}
            \item \texttt{iea37-ex16.yaml} - 16 turbine scenario example layout
            \item \texttt{iea37-ex36.yaml} - 36 turbine scenario example layout
            \item \texttt{iea37-ex64.yaml} - 64 turbine scenario example layout
        \end{itemize}

        \noindent These example layouts are depicted graphically above in \cref{fig:exlayouts}.
        
    \subsubsection{Supplied Code} \label{sec:code}

        To enable participation in this Case Study, we created and supplied a pre-coded Python package.
        This package includes:
            \begin{itemize}
                \item Turbine charactersitics, wind frequency, and wind speed in NREL's \texttt{.yaml} schema described in \cref{sec:filetypes}
                \item Example turbine layouts for each farm size (in \texttt{.yaml} format), displayed graphically in \cref{fig:exlayouts}
                \item Python parsers of the \texttt{.yaml} schema, included in the Appendix
                \item Python target function to calculate AEP (given \texttt{.yaml} turbine locations and farm attributes)
            \end{itemize}

        We selected the programming language Python since it is open source and widely used by researchers in the industry.

        Alteration by the participants to our specific code implementation, or replication of our model in another language is permitted if needed for compatibility with participant optimization methods.
        The alteration is permitted, however, with the understanding that final wind farm layouts will be evaluated with the original python code package that we provided.