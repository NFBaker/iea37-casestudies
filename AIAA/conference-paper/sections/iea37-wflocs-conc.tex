\subsection{Case Study 1}

Results from Case Study 1 show that par4's optimizaton of SNOPT \texttt{+} WEC delivers superior results for the tested wind farms with 16, 36, and 64 turbines.
While information on this method is continuing to be produced, the initial paper written by J.J. Thomas\cite{Thomas2018} describes this method.
Though further testing is required for validation, \textit{par2}'s gradient-free method shows a trend of increased performance that may surpass SNOPT \texttt{+} WEC for wind farms of sizes larger than 64.

\subsection{Case Study 2}

Case study 2 demonstrates that, for wind farms of small area with few turbines, placement on the wind farm boundary delivers superior AEP.
Shortcomings in participant pairings of optimization methods were trapped in local optima, however.
The lesson learned here is to either train researcher intuition to use such layouts as warm starts, or improve optimization methods so that automated optimizers can discover this themselves.
%LES analysis needs to be conducted of participant submissions from Case Study 2, for further conclusions to be drawn.
%\todo[inline]{For the life of me, I can't think of any other points to hit in the conclusion that aren't already covered above. Any suggestions on things to say in this section would be appreciated.}

\subsection{Future Work}\label{ftr-wrk}

\subsubsection{Sample Size}

Though we are happy with the level of participation in the Case Studies, a larger participant sample size with different methods may provide more informative data, or display other novel and superior methods.
To refine our data collection process, we plan on running another round of results for these Case Studies in the near future.

\subsubsection{LES Comparison}

Due to the difficulty in comparability of results between EWMs, we will also run all participant-reported optimized turbine locations through an LES.
With the inherent bias each EWM has for its own optimized locations removed, reported turbine locations will be measured using the same simulation tool for a comparative AEP.
Case Study 2 was constructed mainly for this LES wake model evaluation, in order to gauge which simplified model is most accurate when compared to a higher-cost computational model.
Due to time and computing resource constraints, the authors were unable to run all submitted participant layouts through an LES.

Without this LES analysis, a key piece of adjudication is lacking.
The LES we will use is produced by NREL and called the Simulator fOr Wind Farm Applications (SOWFA)\cite{}.
%Unfortunately, due to the computional time requirements, all participant submissions were not able to be run through SOWFA before the writing of this document.
The LES analysis will be conducted in the near future, and results will be published.