% conclusions:
% - for wind farms of sizes larger than 64. - mention time difference.
% - The conclusions / future work need to be fleshed out a bit more.  It be helpful to comment on things that could have been improved in the way the case study was run, data that was collected, etc.

% appendix:
% - these figures would be better directly labeled rather than using a legend,
% - inconsistent capitalization in references.  Missing doi info.

%\subsection{Case Study 1}
%\label{sec:conc}
We created two case studies to better understand the effects of EWMs and optimization algorithms as applied to the WFLO problem.
Case study 1 focused on optimization methods, and received 10 submissions. 
Case study 2 studied the combination of EWM and optimization method, and received 5 participant submissions.

Results from case study 1 show that \textit{sub4}s use of SNOPT\texttt{+}WEC delivered superior results for the tested wind farms with 16, 36, and 64 turbines.
Although information on this method continues to be produced, the initial paper written by Thomas and Ning \cite{ThomasNing2018} describes this method.
% Time differences displayed in \cref{fig:AEPvsTime} show that \textit{sub4}'s took one of the comparatively longer amounts of time.
% However, this required computational time difference must be counterbalanced with resultant AEP improvement.
% Profitability of which wind farm scenario this method is applied to must be considered to determine if the computational trade-off is worth the extra time it will require.
Regarding \textit{sub2}'s PSQP method, though it showed a trend of increased performance that may surpass SNOPT\texttt{+}WEC for wind farms of sizes larger than 64, further testing is required to validate this pattern.

%\subsection{Case Study 2}

Case study 2 demonstrated that, for wind farms of small area with few turbines, placement on the wind farm boundary delivers superior AEP.
Three of the five participants reported in their cross-comparison that others found superior optima to theirs, indicating that their optimization methods became trapped in a local optima.
It is unclear if the difference was caused by a difference in the optimization approach, or in the wake models' suitability for optimization.  Further investigation is needed in comparing the approaches in more detail.  One approach we would like to pursue is to run all participant-reported optimized turbine locations through a higher-fidelity simulation, like a Large Eddy Simulation.  
% The lesson learned here is to develop better warm start methods, or improve optimization algorithms so that the perturbations do not get trapped in local optima as easily.
%LES analysis needs to be conducted of participant submissions from Case Study 2, for further conclusions to be drawn.
%\todo[inline]{For the life of me, I can't think of any other points to hit in the conclusion that aren't already covered above. Any suggestions on things to say in this section would be appreciated.}

%\subsection{Future Work}\label{sec:ftr-wrk}

%\subsubsection{Sample Size}

Though we are happy with the level of participation in the case studies, a larger participant sample size with different methods may provide more informative data or display other novel and superior methods.
To refine our data collection process, we plan on running another round of results for these case studies in the near future.  In future case studies, more precise wording is needed for questions regarding participant data, especially pertaining to wall time and function calls.  Also, before beginning the case studies, participants should know exactly what information to track and save.

%\subsubsection{LES Comparison}

%An oversight with our questionnaire was that it was the first time some participants knew that number of function calls would be requested.
% For this paper, there were several misunderstandings and errors in data reporting. This was mostly due to imprecise questionnaire wording and not notifying participants what data would be requested until after they had run their optimizations. 
% Some participants had already tracked the necessary data, but those that did not either supplied their best guess, or had to re-run their optimizations to report the data.
% Regarding wall time, some reported the time their best optimization took to run, while others reported how long all 100-1000 optimization runs took together.
% Parallel processing further complicates this metric, so precise wording on how to measure wall time will be needed for future studies.

% Case study 2 was constructed mainly for this LES wake model evaluation in order to gauge which simplified model is most accurate when compared with a higher-cost computational model.
% Due to time and computing resource constraints, the authors were unable to run the submitted participant layouts through an LES.

% Without this LES analysis, a key piece of adjudication is lacking.
% The LES we will use is produced by NREL and called the Simulator fOr Wind Farm Applications (SOWFA)\cite{}.
% %Unfortunately, due to the computional time requirements, all participant submissions were not able to be run through SOWFA before the writing of this document.
% The LES analysis will be conducted in the near future, and results will be published.