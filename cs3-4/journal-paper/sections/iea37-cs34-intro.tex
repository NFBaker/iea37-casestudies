% Intro:
% - your paragraphs in your intro and often overly short.  I’m not really going to go through in detail and suggest where to fix this, but the fact that every paragraph in a whole page in short probably (though not necessarily) suggests that the flow could be better.
% - the literature review is very insufficient.  After the end of the intro you only cited three papers (and one of them is ours).  Certainly we aren’t the only ones who have done relevant stuff in comparing wind farm layout optimization strategies.


\lettrine[nindent=0pt]{T}{he} problem of wind farm layout optimization is multivariate and multimodal.
Inclusion of both many possible inputs (turbine cost, turbine location, wind data granularity, geographic peculiarities, etc.) and many possible target functions (minimize construction costs, maximize energy production, minimize environmental impact, etc.) make the problem as simplistic or as complicated as the user desires.
Computer automated modelling is used to analyze variable permutations and optimize target function calculations on a scale far greater than could be accomplished solely through real-world experimentation.
For these computer optimizations to be successful, two separate factors must be addressed:
\begin{enumerate}
    \item Choice of model inputs and target 
    \item Optimization algorithm selection
\end{enumerate}

Many papers have been written on various turbine and wake models, and even more have been written on optimizations algorithms.
However little has been published on comparative pairings of the two, and how they play out with wind farm characteristics (i.i geography, wind resource, etc.)

We therefore undertook designing a series of simplified wind farm scenarios formed into case studies, where participants would use their model and method pairings to discover as optimal a solution as they were able.
Our intent is to assist researchers in the field by presenting a standardized method of comparing widn farm layout optimization methods.
Case studies 1 and 2 (cs1 \& cs2) were wind farm scenarios with circular boundaries of three differnt sizes to observe patterns over increasing complexity.
Case studies 3 and 4 (cs2 \& cs4) introduced both non-uniform boundaries and a more complicated wind resource than cs1 \& cs2, to further test participant methods.

\textcolor{red}{Mention results here}

%-----------------------%