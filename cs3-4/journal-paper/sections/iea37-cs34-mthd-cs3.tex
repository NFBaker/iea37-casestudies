Mirrorring cs1, the purpose of cs3 was to isolate optimization practices for WFLO with boundaries having concavities.
To isolate this variable, we mandated that participants use a single representative EWM.
We selected a generalized EWM that could be used by both gradient-based and gradient-free optimization algorithms, that was also computationally inexpensive in comparison to LES and DNS methods.
    
    \subsubsection{Wake Model} \label{sec:wakemodel}
        The wake model selected for cs3 is the same used in cs1, a simplified version of Bastankhah's Gaussian wake model \cite{Bastankhah2016,ThomasNing2018}.
        This wake model is described \cref{eq:wakemodel}.
%
        \begin{equation}
            \frac{\Delta U}{U_{\infty}}
            =
            \Bigg(
            1 - \sqrt{
                1 - \frac{C_T}
                {8\sigma_{y}^{2}/D^2}
            }
            \Bigg)
            \text{exp}\bigg(
            -0.5\Big(
                \frac{y-\delta}{\sigma_{y}}
                \Big)^2
            \bigg)
            \label{eq:wakemodel}
        \end{equation}
%
        In \cref{eq:wakemodel}, $\Delta U/U_{\infty}$ is the wake velocity deficit, $C_T = 8/9$ is the thrust coefficient, $y-\delta$ is the distance of the point of interest from the wake center in the cross-stream horizontal direction, $D$ is the turbine diameter, and $\sigma_y$ is the standard deviation of the wake spread in the cross-stream horizontal direction as defined in \cref{eq:sigy}:

        \begin{equation}
            \sigma_y = (k_y x) + \frac{D}{\sqrt{8}} \\
            \label{eq:sigy}
        \end{equation}

        In \cref{eq:sigy}, $x$ is the downstream distance from the turbine generating the wake to the turbine of interest, and $D$ is the turbine diameter. The variable $k_y$ is determined as a function of turbulence intensity ($I$).
        In this case study turbulence intensity was treated as a constant of $0.075$, and we therefore used a corresponding $k_{y}$ of $0.0324555$ \cite{Niayifar2016,ThomasNing2018}.

        Increasing turbulence intensity has numerous effects and draws attention away from the main purpose of these case studies, which was to observe the differences of optimization strategies.
        For the wake model we used (shown in \cref{eq:wakemodel}), increasing the turbulence intensity widened the wake cone, but second and third order effects are unknown.
        As such, we used a very low intensity in an attempt to minimize the considered variables.

	\subsubsection{Number of Turbines}
        Though cs1 \& cs2 used multiple farm sizes to observe the effects of scaling, the focus of cs3 is to observe methods regarding the unique boundary geography.
        For this reason, only a single sized farm is used in cs3, consisting of twenty-five (25) turbines.
        Like cs1 \& cs2, the number of turbines for cs3 \& cs4 were selected as perfect squares.
        
    \subsubsection{Supplied Code} \label{sec:code}

        We provided participants with a link to a GitHub repository\footnote{https://github.com/byuflowlab/iea37-wflo-casestudies/cs2-4/} which included files with the following contents:
            \begin{itemize}
                \item Turbine characteristics, wind frequency, and wind speed in IEA 37's \texttt{.yaml} schema
                \item Boundary coordinates and an example turbine layout (in \texttt{.yaml} format)
                \item Python parsers of the \texttt{.yaml} schema
                \item Python target function to calculate AEP (given \texttt{.yaml} turbine locations and farm attributes)
            \end{itemize}
        We selected the programming language Python, since it is widely used by researchers in the industry, and is open source.
        Participants were allowed to alter our specific code implementation or replicate the provided model in another language to speed up the code or for compatibility with their optimization methods. 
        This was with the understanding, however, that final wind farm layouts would be evaluated with the original Python code that we provided.