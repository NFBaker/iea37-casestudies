The intent of this case study was to assess both
(1) the effects different wake models have on optimization results and
(2) how participants addressed a farm with discreet boundary sections seperated by non-viable areas.
Like the relationship between cs2 and cs1, cs4 differs from cs3 in that no wake model was provided.
Participants were free to choose their preferred EWM and optimization method combination.

% \subsubsection{Wake Model}
% Unlike cs3, participant-reported AEP values are not comparable, since each participant used a different EWMs to calculate AEP. %(which account for different physics phenomena)
% To help us make fair comparisons and conclusions, we conducted a cross-comparison of results between participants.
% For the cross-comparison, each participant provided their optimal turbine layout in the standardized \texttt{.yaml} format. Each participant was then provided with every other participants' optimized layout file.
% Participants then used their own wake model to calculate the AEP of the other participant's proposed farm layouts with their EWMs.
% From this portion of the case study, we hoped to learn if any participants' results were consistently seen as superior by other EWMs.

%\subsubsection{Farm Attributes}
The wind farm in cs4 consisted of eighty-one (81) turbines.
It was entirely up to participants and their optimizers how to apportion the turbines across the disparate sections of the farm.
The wind rose and wind speed frequencies were the same as in cs3.